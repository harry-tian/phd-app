
% generic
All my experiences collectively shaped my research interests and motivated me to enroll in a PhD program, where I can further develop my research career and pursue my goal of building more effective and responsible human-AI collaboration. 
% specific
University of Utah's PhD program in Computer Science is appealing to me because of the department's strong AI and HCI research and the many faculty members whose interests align with mine. 
% faculty 1
I am interested in Professor Ana Marasović's work on explainable and interpretable AI. I was inspired her work \textit{Formalizing Trust in Artificial Intelligence: Prerequisites, Causes and Goals of Human Trust in AI} [\citenum{Jacovi2021}]. In ML and HAI research, many terms and concepts, e.g., "trust", "complementarity", are frequently used without clear definitions, so formalizing those concepts is a important and fundamental step. 
I like their definition of human-AI trust as it emphasizes vulnerability and warranted trust, which are prevalent factors in real-life, high risk task settings. On the other hand, this also points to flaws in HAI lab-based research on non-expert laypeople: there is little risk in the experiment setting so they would not feel vulnerable to the AI; laypeople's lack of risk and lack domain knowledge could also lead to unwarranted trust. I would be excited to leverage this human-AI trust framework to bridge the between laypeople and real-world practitioners in HAI experiments.
% faculty 2
I am also interested in Professor Ellen Riloff’s direction on sentiment analysis.
% conclusion
My interests in HAI are well represented at University of Utah, and I believe a PhD in Computer Science will allow me to build more effective and responsible human-AI collaboration.