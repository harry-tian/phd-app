


% generic
All my experiences collectively shaped my research interests and motivated me to enroll in a PhD program, where I can further develop my research career and pursue my goal of building more effective and responsible human-AI collaboration. 
% specific
University of Texas at Austin's PhD in IROM is appealing to me because of the department's strong AI and HCI research and the many faculty members whose interests align with mine.
% faculty 1
I am especially interested in Professor Maria De-Arteaga's research on AI-assisted decision support, a direction that overlaps with mine. I enjoyed her paper \textit{A Case for Humans-in-the-Loop: Decisions in the Presence
of Erroneous Algorithmic Scores} as the analysis on people use AFST, though retrospective, was very thorough [\citenum{De-Arteaga2020}]. I also think results that call workers make use of other information in addition to risk assessment and can override wrong decisions point to significant influence of task expertise in human-algorithm collaboration. Additionally, I really enjoyed her work on unobservables in HAI [\citenum{Holstein2022}]. That inputs to AI are much more restricted than information available to humans is a well established concept, but leveraging this for human-AI complementarity is novel. So I realized that complementarity is not restricted to just performance and I would be excited to explore more forms of human-AI complementarity.
% faculty 2
I am also interested in Professor Maytal Saar-Tsechansky's direction in trustworthy human-AI collaboration. 
% conclusion
My interests in HAI are well represented at University of Texas at Austin, and I believe a PhD in IROM will allow me to build more effective and responsible human-AI collaboration.