
% generic
All my experiences collectively shaped my research interests and motivated me to enroll in a PhD program, where I can further develop my research career and pursue my goal of building more effective and responsible human-AI collaboration. 
% specific
Harvard University's PhD program in Computer Science is appealing to me because of the department's strong AI and HCI research and the many faculty members whose interests align with mine. 
% faculty 1
I attended the Human+AI Conference at UChicago and was inspired by Professor Krzysztof Gajos' talk. I learned that in human-AI collaboration, cognitive engagement and human psychological factors are equally, if not more, important than AI models and explanations. This further led to my realization that, we HAI researchers often claim humans use explanation in “unexpected” ways, but this is because we are making wrong assumptions about humans; instead human-AI systems should account for how humans would behave in the presence of AI. Also, related to his work on cognitive engagement [\citenum{bucinca2021}], I am curious about the effects of “voluntary assistance”, where AI assistance is hidden from humans but will be shown upon request, on reducing cognitive work and increasing cognitive engagement. 
% faculty 2
I am also interested in Professor Elena Glassman’s research on HAI. I liked her work on proxy tasks and subjective measures [\citenum{bucinca2020}], as it points to the misalignment between HAI lab research and real-life settings in task and measurement. 
% faculty 3
I am also interested in Professor Finale Doshi-Velez’s direction on HAI and interpretable ML. 
% conclusion
My interests in HAI are well represented at Harvard, and I believe a PhD in Computer Science will allow me to build more effective and responsible human-AI collaboration.