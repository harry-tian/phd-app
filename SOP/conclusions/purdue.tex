


% generic
All my experiences collectively shaped my research interests and motivated me to enroll in a PhD program, where I can further develop my research career and pursue my goal of building more effective and responsible human-AI collaboration. 
% specific
Purdue University's PhD in CS is appealing to me because of the department's strong AI and HCI research and the many faculty members whose interests align with mine.
% faculty 1
I am especially interested in Professor Ming Yin's research on HAI. I enjoyed her paper \textit{Will You Accept the AI Recommendation? Predicting Human Behavior in AI-Assisted Decision Making} [\citenum{willyou}], as it is similar in spirit to my project on case-based decision support [\citenum{human-compat}] in that the two papers both leverage ML models that learn from human behavior data for AI-assisted decision-making. I also liked her paper \textit{You’d Better Stop! Understanding Human Reliance on Machine Learning Models under Covariate Shift} as I like the direction of communicating and educating ML models to end-users [\citenum{youbetterstop}]. I would be excited to work on this direction by incorporating machine teaching and example-based explanations; for instance, we can choose more representative prototypes and criticisms for ML model.
% faculty 2
I am also interested in Professor Tianyi Zhang's direction human-centered softwares and Professor Dan Goldwasser's direction work in connecting natural language with real world scenarios.
% conclusion
My interests in HAI are well represented at Purdue University, and I believe a PhD in CS will allow me to build more effective and responsible human-AI collaboration.