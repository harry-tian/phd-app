%!TEX root = main.tex

\sect{interest and motivation}



\subsect{interest}
At UChicago, through Dr. Chenhao Tan's course in Human-Center Machine Learning I learned of the complexities of human-AI interaction. In high-stake domains like medical diagnosis, full automation of AI is often not desired and humans have to make the final decision. I am intrigued by the problems caused by this limitation. For example, for tasks where AI models outperform humans, many AI assistance and explanations improve human performance at the cost of humans blindly following the AI's decision. On the AI side, can we design neutral, unpersuasive AI assistance that retains human agency while improving performance? On the human side, can we train humans to judge AI assistance with more scrutiny. My past research aimed to answer these questions and I plan to continue solving these questions in my future research.
\\

\subsect{direct impact}
Also, I am motivated by the direct and visible impact in human-centered AI research. My research project [citation] has improved crowdworkers' performance on pneumonia diagnosis in chest X-rays, meaning it has the potential to be implemented in real-life medical settings. Such human-related experiments and improvements are very rewarding to me.
% \\ 


\subsect{interdisciplinary}
Finally I enjoy the interdisciplinary aspects of human-centered AI research. 
My past research in human-compatible AI decision-support involved modeling human perception, so we used triplet annotations, a method from experimental psychology. 
My current research aims to leverage our human-compatible AI build radiology teaching framework and we are exploring psychology literature in learning and categorization. 
More generally, human-centered AI revolves around how humans interact with a decision-making entity and thus involves many many different fields like economics, sociology, ethics, legal, etc. 
An exciting aspect of human-centered AI resarch is learning and utilizing knowledge from multiple fields. 
\\



% Different from conventional technology that are mostly transparent and in full human control, current AI models are black-box decision-making entities that have may significant influences in humans' decision making, but the influences may not always be positive. In high-stake domains like medical diagnosis, complete reliance on AI decisions may not be desirable. Thus numerous issues arise, like generating AI assistance to inspire appropriate humans trust and reliance on AI, retaining human agency, how to properly explain model decisions, etc. I was intrigued by the plethora of complexities raised by the differences and incompatibilities between humans and AI. 

% My research in particular sparked my interest in how to design AI assistance to inspire appropriate human trust and reliance on the AI while simultaneously improving human performance.

% \sect{interest in AI}
% My interest in the AI traces to my interest in computer science. I was amazed by what programming and algorithms can achieve and how they shape literally everything in the modern world. I was even more baffled when I learned of machine learning models; I am still fascinated by neutral networks' learning ability that arises from simple matrix transformations and gradient updates. 

% \sect{interest in research}
% My interest in human-centered research can be attributed to the many social science and history courses I took as well as a childhood dream of doing good for mankind. I was never satisfied working solely on the technical, mathematical details of CS; I always aspired tackling issues that were more human-related have more direct and immediate social, "visible" impact. I still remember getting excited when attending a researcher talked about her projects that helped blind children learn in non-blind schools and an interactive projector that helped separated families play games on a table.

% \sect{interest in interdisciplinary research}
% I also greatly enjoy the interdisciplinary aspects of human-centered AI research. As a Computer Science and Linguistics double major, my initial interest in ML was in NLP as I was excited to connect knowledge from both fields. I fulfilled my interdisciplinary passion as my past research in human-compatible AI decision-support involved modeling human perception, an issue explored by experimental and cognitive psychologists. My current research aims to leverage our human-compatible AI to provide more effective teaching frameworks for radiology residents and we are exploring psychology literature in teaching and learning. On a broader scale, human-centered AI revolves around how humans interact with a decision-making entity and thus involves many many different fields like economics, sociology, ethics, legal, etc.
