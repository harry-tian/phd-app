%!TEX root = main.tex

\sect{development of research experience}

In our project, we spent a large amount of time devising a decision support system that differed from existing literature providing model explanations: our goal was to provide faithful and neutral evidence for humans to make independent decisions. Thus we eventually devised a neutral decision support policy that did not reveal AI's predicted label and provided nearest-neighbor explanations from all classes. 
We showed such a policy provided effective decision support, but it was less effective than a persuasive decision support policy closer to model explanations. 

This alludes to a larger problem: the tradeoff between human agency and human-AI team performance. 
Many human-AI teams provide humans with "persuasive" AI assistance that improves humans performance but at the expense of humans' overreliance and blind trust towards AI [\citenum{bansal2021}]. This is undesired and unethical, especially in high-stake domains where humans should have the last say.
On the other hand, many work [\citenum{bansal2021}], including ours, also show that neutral, non-persuasive assistance that dissuade humans from blindly following AI perform no better than persuasive assistance. 
Thus, I am interested in solving this dilemma.


% \subsect{class that motivated me}
% At UChicago, through Dr. Chenhao Tan's course in Human-Center Machine Learning I learned of the complexities of human-AI interaction. In high-stake domains like medical diagnosis, full automation of AI is often not desired and humans have to make the final decision. I am intrigued by the problems caused by this limitation, such as generating explanations that are informative and neutral instead of persuasive and deceiving, complementary performance etc. 

% \subsect{direct impact}
% Also, I am motivated by the direct and visible impact in human-centered AI research. My research project [\citenum{human-compat}] has improved crowdworkers' performance on pneumonia diagnosis in chest X-rays, meaning it has the potential to be implemented in real-life medical settings. Such human-related experiments and improvements are very rewarding to me.
% % \\ 


% \subsect{interdisciplinary}
% Finally I enjoy the interdisciplinary aspects of human-centered AI research. 
% My past research in human-compatible AI decision-support involved modeling human perception, so we used triplet annotations, a method from experimental psychology. 
% My current research aims to leverage our human-compatible AI build radiology teaching framework and we are exploring psychology literature in learning and categorization. 
% More generally, human-centered AI revolves around how humans interact with a decision-making entity and thus involves many many different fields like economics, sociology, ethics, legal, etc. 
% An exciting aspect of human-centered AI resarch is learning and utilizing knowledge from multiple fields. 
% \\



