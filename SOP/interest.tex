%!TEX root = main.tex

\section{general research interest \& motivation}


% \subsection{interest in AI}
% My interest in the AI traces to my interest in computer science. I was amazed by what programming and algorithms can achieve and how they shape literally everything in the modern world. I was even more baffled when I learned of machine learning models; I am still fascinated by neutral networks' learning ability that arises from simple matrix transformations and gradient updates. 

\subsection{interest in research}
My interest in human-centered research can be attributed to the many social science and history courses I took as well as a childhood dream of doing good for mankind. I was never satisfied working solely on the technical, mathematical details of CS; I always aspired tackling issues that were more human-related have more direct and immediate social, "visible" impact. I still remember getting excited when attending a researcher talked about her projects that helped blind children learn in non-blind schools and an interactive projector that helped separated families play games on a table.

\subsection{interest in HAI}
% Human-centered AI perfectly connects my interest in AI and HCI. What makes human-AI interaction especially interesting to me is that, different from conventional technology and algorithms that humans would just use, AI models are decision-making entities (think of the cliche AI-taking-over-the-world scenario). This sparks a plethora of uncertainties and issues, like humans' trust and reliance on AI decisions, who is responsible for AI decisions, when and how should we follow them, how we explain them, etc. I am interested in framing appropriate AI assistance and human-AI interactions that retain human agency while optimizing human-AI team performance. 

My interest in HAI is largely inspired by my research with Dr. Chenhao Tan at the University of Chicago as well as his course in Human-Center Machine Learning. Different from conventional technology that are mostly transparent and in full human control, current AI models are black-box decision-making entities that have may significant influences in humans' decision making, but the influences may not always be positive. In high-stake domains like medical diagnosis, complete reliance on AI decisions may not be desirable. Thus numerous issues arise, like generating AI assistance to inspire appropriate humans trust and reliance on AI, retaining human agency, how to properly explain model decisions, etc. I was intrigued by the plethora of complexities raised by the differences and incompatibilities between humans and AI. I believe human-AI interaction is an important direction of research. 


% My research in particular sparked my interest in how to design AI assistance to inspire appropriate human trust and reliance on the AI while simultaneously improving human performance.

\subsection{interest in interdisciplinary research}
I also greatly enjoy the interdisciplinary aspects of human-centered AI research. As a Computer Science and Linguistics double major, my initial interest in ML was in NLP as I was excited to connect knowledge from both fields. I fulfilled my interdisciplinary passion as my past research in human-compatible AI decision-support involved modeling human perception, an issue explored by experimental and cognitive psychologists. My current research aims to leverage our human-compatible AI to provide more effective teaching frameworks for radiology residents and we are exploring psychology literature in teaching and learning. On a broader scale, human-centered AI revolves around how humans interact with a decision-making entity and thus involves many many different fields like economics, sociology, ethics, legal, etc.
